\chapter{结论}

\section{本系统的不足之处}

由于本系统采用了Flask+MySQL的技术路线,因此疫情数据并不会实时地更新,只有在更新数据库且刷新网页之后才会更新疫情数据。针对此问题,作者正在学习nginx与Ajax的相关知识,目的在于后期改进本系统,以实现疫情数据实时更新的功能。实时更新所带来的问题就是如何处理MySQL的高并发问题。\citep{翁志宁2016flask}针对此问题,可以对数据库字段、索引进行优化,解藕模块。\citep{牛作东2019基于}

\section{所学到的知识}

通过本研究,掌握了MySQL数据库的基本操作。通过爬取腾讯疫情数据,学会了如何使用网页前端的调试工具,如何通过Python分析json数据。通过pymysql创建数据库连接,将字典中数据插入到数据表中,使得Python,MySQL,网页能够联动起来,形成一个轻量级的Web应用。这种轻量级应用部署简单,运行可靠稳定,且现实意义较大,符合当今疫情防控常态化的发展趋势。\citep{虞乔木2020新冠肺炎疫情防控常态化研究}





