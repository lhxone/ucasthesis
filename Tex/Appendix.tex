\chapter{Python代码}

\section{spider.py---用来爬取网页数据}

\lstset{
 columns=fixed,       
 numbers=left,                                        % 在左侧显示行号
 numberstyle=\tiny\color{gray},                       % 设定行号格式
 frame=none,                                          % 不显示背景边框
 backgroundcolor=\color[RGB]{245,245,244},            % 设定背景颜色
 keywordstyle=\color[RGB]{40,40,255},                 % 设定关键字颜色
 numberstyle=\footnotesize\color{darkgray},           
 commentstyle=\it\color[RGB]{0,96,96},                % 设置代码注释的格式
 stringstyle=\rmfamily\slshape\color[RGB]{128,0,0},   % 设置字符串格式
 showstringspaces=false,                              % 不显示字符串中的空格
 language=Python,                                        % 设置语言
}
\begin{lstlisting}
import os
import pymysql
import time
import json
import traceback  # 追踪异常
import requests


def get_tencent_data():
    url = 'https://view.inews.qq.com/g2/getOnsInfo?name=disease_h5'
    url_his = 'https://view.inews.qq.com/g2/getOnsInfo?name=disease_other'
    headers = {
        'user-agent': 'Mozilla/5.0 (Windows NT 10.0; Win64; x64) AppleWebKit/537.36 (KHTML, like Gecko) Chrome/65.0.3325.181 Safari/537.36',
    }

    r = requests.get(url, headers)
    res = json.loads(r.text)  # json字符串转字典
    data_all = json.loads(res['data'])

    r_his = requests.get(url_his, headers)
    res_his = json.loads(r_his.text)
    data_his = json.loads(res_his['data'])

    history = {}  # 历史数据

    for i in data_his["chinaDayList"]:
        ds = "2020." + i["date"]
        tup = time.strptime(ds, "%Y.%m.%d")
        ds = time.strftime("%Y-%m-%d", tup)  # 改变时间格式
        confirm = i["confirm"]
        suspect = i["suspect"]
        heal = i["heal"]
        dead = i["dead"]
        history[ds] = {"confirm": confirm, "suspect": suspect, "heal": heal, "dead": dead}
    for i in data_his["chinaDayAddList"]:
        ds = "2020." + i["date"]
        tup = time.strptime(ds, "%Y.%m.%d")
        ds = time.strftime("%Y-%m-%d", tup)
        confirm = i["confirm"]
        suspect = i["suspect"]
        heal = i["heal"]
        dead = i["dead"]
        history[ds].update({"confirm_add": confirm, "suspect_add": suspect, "heal_add": heal, "dead_add": dead})

    # 下面就不用动了
    details = []  # 当日详细数据
    update_time = data_all["lastUpdateTime"]
    data_country = data_all["areaTree"]  # list 25个国家
    data_province = data_country[0]["children"]  # 中国各省
    for pro_infos in data_province:
        province = pro_infos["name"]  # 省名
        for city_infos in pro_infos["children"]:
            city = city_infos["name"]
            confirm = city_infos["total"]["confirm"]
            confirm_add = city_infos["today"]["confirm"]
            heal = city_infos["total"]["heal"]
            dead = city_infos["total"]["dead"]
            details.append([update_time, province, city, confirm, confirm_add, heal, dead])
    return history, details


def get_conn():
    """
    :return: 连接,游标
    """
    # 创建连接
    conn = pymysql.connect(host="127.0.0.1",
                           user="root",
                           password="123456",
                           db="cov",
                           charset="utf8")
    # 创建游标
    cursor = conn.cursor()  # 执行完毕返回的结果集默认以元组显示
    return conn, cursor


def close_conn(conn, cursor):
    if cursor:
        cursor.close()
    if conn:
        conn.close()


def update_details():
    """
    更新 details 表
    :return:
    """
    cursor = None
    conn = None
    try:
        li = get_tencent_data()[1]  # 0 是历史数据字典,1 最新详细数据列表
        conn, cursor = get_conn()
        sql = """insert into details(update_time,province,city,confirm,confirm_add,heal,dead) values(%s,%s,%s,%s,%s,%s,%s)"""
        sql_query = """select %s='(select update_time from details order by id desc limit 1)'"""  # 对比当前最大时间戳
        cursor.execute(sql_query, li[0][0])
        if not cursor.fetchone()[0]:
            print(f"{time.asctime()}开始更新最新数据")
            for item in li:
                cursor.execute(sql, item)
            conn.commit()  # 提交事务 update delete insert操作
            print(f"{time.asctime()}更新最新数据完毕")
        else:
            print(f"{time.asctime()}已是最新数据!")
    except:
        traceback.print_exc()
    finally:
        close_conn(conn, cursor)


def insert_history():
    """
        插入历史数据
    :return:
    """
    cursor = None
    conn = None
    try:
        dic = get_tencent_data()[0]  # 0 是历史数据字典,1 最新详细数据列表
        print(f"{time.asctime()}开始插入历史数据")
        conn, cursor = get_conn()
        sql = """insert into history values(%s,%s,%s,%s,%s,%s,%s,%s,%s)"""
        for k, v in dic.items():
            # item 格式 {'2020-01-13': {'confirm': 41, 'suspect': 0, 'heal': 0, 'dead': 1}
            cursor.execute(sql, [k, v.get("confirm"), v.get("confirm_add"), v.get("suspect"),
                                 v.get("suspect_add"), v.get("heal"), v.get("heal_add"),
                                 v.get("dead"), v.get("dead_add")])

        conn.commit()  # 提交事务 update delete insert操作
        print(f"{time.asctime()}插入历史数据完毕")
    except:
        traceback.print_exc()
    finally:
        close_conn(conn, cursor)


def update_history():
    """
    更新历史数据
    :return:
    """
    cursor = None
    conn = None
    try:
        dic = get_tencent_data()[0]  # 0 是历史数据字典,1 最新详细数据列表
        print(f"{time.asctime()}开始更新历史数据")
        conn, cursor = get_conn()
        sql = """insert into history (ds,confirm,suspect,heal,dead,confirm_add,suspect_add,heal_add,dead_add) values(%s,%s,%s,%s,%s,%s,%s,%s,%s)"""
        sql_query = """select confirm from history where ds=%s"""
        for k, v in dic.items():
            # item 格式 {'2020-01-13': {'confirm': 41, 'suspect': 0, 'heal': 0, 'dead': 1}}
            if not cursor.execute(sql_query, k):
                cursor.execute(sql, [k, v.get("confirm"), v.get("confirm_add"), v.get("suspect"),
                                     v.get("suspect_add"), v.get("heal"), v.get("heal_add"),
                                     v.get("dead"), v.get("dead_add")])
        conn.commit()  # 提交事务 update delete insert操作
        print(f"{time.asctime()}历史数据更新完毕")
    except:
        traceback.print_exc()
    finally:
        close_conn(conn, cursor)


if __name__ == "__main__":
    insert_history()
    update_history()
    update_details()
\end{lstlisting}

\section{utils.py---app.py中所调用的方法}

\begin{lstlisting}
    import time
    import pymysql
    from decimal import Decimal
    import json
    
    
    def get_time():
        time_str = time.strftime("%Y{}%m{}%d{} %X")
        return time_str.format("年", "月", "日")
    
    
    def get_conn():
        """
        :return: 连接,游标
        """
        # 创建连接
        conn = pymysql.connect(host="localhost",
                               user="root",
                               password="123456",
                               db="cov",
                               charset="utf8")
        # 创建游标
        cursor = conn.cursor()  # 执行完毕返回的结果集默认以元组显示
        return conn, cursor
    
    
    def close_conn(conn, cursor):
        cursor.close()
        conn.close()
    
    
    def query(sql, *args):
        """
        封装通用查询
        :param sql:
        :param args:
        :return: 返回查询到的结果,((),(),)的形式
        """
        conn, cursor = get_conn()
        cursor.execute(sql, args)
        res = cursor.fetchall()
        close_conn(conn, cursor)
        return res
    
    
    def get_c1_data():
        """
        :return: 返回大屏div id=c1 的数据
        """
        # 因为会更新多次数据,取时间戳最新的那组数据
        sql = "select sum(confirm)," \
              "(select suspect from history order by ds desc limit 1)," \
              "sum(heal)," \
              "sum(dead) " \
              "from details " \
              "where update_time=(select update_time from details order by update_time desc limit 1) "
              #"where update_time=(select update_time from details order by update_time desc limit 1) "
        res = query(sql)
    
        res_list = [str(i) for i in res[0]]
        res_tuple = tuple(res_list)
        return res_tuple
    
    
    def get_c2_data():
        """
        :return:  返回各省数据
        """
        # 因为会更新多次数据,取时间戳最新的那组数据
        sql = "select province,sum(confirm) from details " \
              "where update_time=(select update_time from details " \
              "order by update_time desc limit 1) " \
              "group by province"
        res = query(sql)
        return res
    
    
    def get_l1_data():
        sql = "select ds,confirm,suspect,heal,dead from history"
        res = query(sql)
        return res
    
    
    def get_l2_data():
        sql = "select ds,confirm_add,suspect_add from history"
        res = query(sql)
        return res
    
    
    def get_r1_data():
        """
        :return:  返回非湖北地区城市确诊人数前5名
        """
        sql = 'SELECT province,confirm_add FROM ' \
              '(select province,sum(confirm_add) as confirm_add from details  ' \
              'where update_time=(select update_time from details order by update_time desc limit 1) ' \
              'group by province) as a ' \
              'ORDER BY confirm_add DESC LIMIT 5'
        res = query(sql)
        return res
    
    
    def get_r2_data():
        '''
            获取世界各国的疫情数据
            :return:
        '''
        # 因为会更新多次数据,取时间戳最新的那组数据
        sql = "select province,sum(confirm_add) from details " \
              "where update_time=(select update_time from details " \
              "order by update_time desc limit 1) " \
              "group by province"
        res = query(sql)
        return res
    
    
    if __name__ == "__main__":
        print(get_c1_data())
\end{lstlisting}    

\section{app.py---启动Flask的主程序}

\begin{lstlisting}
    import utils
    from flask import Flask
    from flask import request
    from flask import render_template
    from flask import jsonify
    from flask_bootstrap import Bootstrap
    import os
    import string
    
    os.system(
        'python3 /Users/liuhaoxin/Documents/GitHub/covid-19-china/spider.py && python3 /Users/liuhaoxin/Documents/GitHub/covid-19-china/utils.py')
    
    app = Flask(__name__)
    bootstrap = Bootstrap(app)
    
    @app.route('/')
    def hello_world():
        return render_template("index.html")
    
    
    @app.route("/c1")
    def get_c1_data():
        data = utils.get_c1_data()
        return jsonify({"confirm":data[0],"suspect":data[1],"heal":data[2],"dead":data[3]})
    
    
    @app.route("/c2")
    def get_c2_data():
        res = []
        for tup in utils.get_c2_data():
            # print(tup)
            res.append({"name":tup[0],"value":int(tup[1])})
        return jsonify({"data":res})
    
    
    @app.route("/l1")
    def get_l1_data():
        data = utils.get_l1_data()
        day,confirm,suspect,heal,dead = [],[],[],[],[]
        for a,b,c,d,e in data[7:]:
            day.append(a.strftime("%m-%d"))
            #  a是datatime类型
            confirm.append(b)
            suspect.append(c)
            heal.append(d)
            dead.append(e)
        return jsonify({"day": day, "confirm": confirm, "suspect": suspect, "heal": heal, "dead": dead})
    
    
    @app.route("/l2")
    def get_l2_data():
        data = utils.get_l2_data()
        day, confirm_add, suspect_add = [], [], []
        for a, b, c in data[7:]:
            day.append(a.strftime("%m-%d"))  # a是datatime类型
            confirm_add.append(b)
            suspect_add.append(c)
        return jsonify({"day": day, "confirm_add": confirm_add, "suspect_add": suspect_add})
    
    
    @app.route("/r1")
    def get_r1_data():
        data = utils.get_r1_data()
        city = []
        confirm = []
        for k,v in data:
            city.append(k)
            confirm.append(int(v))
        return jsonify({"city": city, "confirm": confirm})
    
    
    @app.route("/r2")
    def get_r2_data():
        res = []
        for tup in utils.get_r2_data():
            # print(tup)
            res.append({"name": tup[0], "value": int(tup[1])})
        return jsonify({"data": res})
    
    
    @app.route("/time")
    def get_time():
        return utils.get_time()
    
    
    if __name__ == '__main__':
        # app.run(host='127.0.0.1', port=5000)
        bootstrap.run()
\end{lstlisting}

\chapter{网页代码}

\section{index.html}

\lstset{
 columns=fixed,       
 numbers=left,                                        % 在左侧显示行号
 numberstyle=\tiny\color{gray},                       % 设定行号格式
 frame=none,                                          % 不显示背景边框
 backgroundcolor=\color[RGB]{245,245,244},            % 设定背景颜色
 keywordstyle=\color[RGB]{40,40,255},                 % 设定关键字颜色
 numberstyle=\footnotesize\color{darkgray},           
 commentstyle=\it\color[RGB]{0,96,96},                % 设置代码注释的格式
 stringstyle=\rmfamily\slshape\color[RGB]{128,0,0},   % 设置字符串格式
 showstringspaces=false,                              % 不显示字符串中的空格
 language=html,                                        % 设置语言
}
\begin{lstlisting}
    <!DOCTYPE html>
    <html>
    <head>
        <meta charset="utf-8">
        <meta http-equiv="X-UA-Compatible" content="IE=edge">
        <meta name="viewport" content="width=device-width, initial-scale=1.0">
        <title>疫情监控系统</title>
        <link rel="icon" href="../static/icon.ico">
        <link rel="stylesheet" href="https://cdn.staticfile.org/twitter-bootstrap/3.3.7/css/bootstrap.min.css">
        <script src="https://cdn.staticfile.org/jquery/2.1.1/jquery.min.js"></script>
        <script src="https://cdn.staticfile.org/twitter-bootstrap/3.3.7/js/bootstrap.min.js"></script>
        <script src="../static/js/jquery-1.11.1.min.js"></script>
        <script src="../static/js/echarts.min.js"></script>
        <script src="../static/js/china.js"></script>
        <link href="../static/css/main.css" rel="stylesheet"/>
    </head>
    <body>
    
    <div id="tim"></div>
    <div id="gyroContain" class="col clearfix">
        <div class="row clearfix">
            <div class="col-md-12 column">
                <nav class="navbar navbar-default" role="navigation">
                    <div class="navbar-header">
                        <a class="navbar-brand" href="#">疫情数据监控平台</a>
                    </div>
    
                    <div class="collapse navbar-collapse" id="bs-example-navbar-collapse-1">
                        <ul class="nav navbar-nav">
                            <li class="active">
                                <a href="#">国内数据</a>
                            </li>
                            <li>
                                <a href="#">国外数据</a>
                            </li>
                            <li class="dropdown">
                                <a href="#" class="dropdown-toggle" data-toggle="dropdown">更多<strong
                                        class="caret"></strong></a>
                                <ul class="dropdown-menu">
                                    <li>
                                        <a href="https://www.baidu.com/#ie=utf8&wd=疫情最新政策">最新政策</a>
                                    </li>
                                    <li>
                                        <a href="https://www.baidu.com/#ie=utf8&wd=疫情当地政策">当地政策</a>
                                    </li>
                                    <li>
                                        <a href="https://www.baidu.com/#ie=utf8&wd=疫情国际消息">国际消息</a>
                                    </li>
                                    <li class="divider">
                                    </li>
                                    <li>
                                        <a href="#">健康码</a>
                                    </li>
                                    <li class="divider">
                                    </li>
                                    <li>
                                        <a href="#">防疫平台</a>
                                    </li>
                                </ul>
                            </li>
                        </ul>
    
                        <ul class="nav navbar-nav navbar-right">
                            <form class="navbar-form navbar-left" role="search">
                                <div class="form-group">
                                    <input type="text" class="form-control"/>
                                </div>
                                <button type="submit" class="btn btn-default">搜索</button>
                            </form>
                        </ul>
                    </div>
    
                </nav>
            </div>
        </div>
        <div class="row-cols-md-4 row">
            <div id="l1">我是左1</div>
            <div id="l2">我是左2</div>
        </div>
        <div class="row-cols-md-4 row">
            <div id="c1" class="row clearfix">
                <div class="txt"><h2 class="text-lg-center text-center">累计确诊</h2></div>
                <div class="txt"><h2 class="text-lg-center text-center">剩余疑似</h2></div>
                <div class="txt"><h2 class="text-lg-center text-center">累计治愈</h2></div>
                <div class="txt"><h2 class="text-lg-center text-center">累计死亡</h2></div>
                <div class="num"><h1 class="text-primary text-center"></h1></div>
                <div class="num"><h1 class="text-primary text-center"></h1></div>
                <div class="num"><h1 class="text-primary text-center"></h1></div>
                <div class="num"><h1 class="text-primary text-center"></h1></div>
            </div>
            <div id="c2">我是中2</div>
        </div>
        <div class="row-cols-md-4 row">
            <div id="r1">我是右1</div>
            <div id="r2">我是右2</div>
        </div>
    
        <script src="../static/js/ec_center.js"></script>
        <script src="../static/js/ec_left1.js"></script>
        <script src="../static/js/ec_left2.js"></script>
        <script src="../static/js/ec_right1.js"></script>
        <script src="../static/js/ec_right2.js"></script>
        <script src="../static/js/controller.js"></script>
    </div>
    </body>
    </html>
\end{lstlisting}

\section{controller.js}
\begin{lstlisting}
    function gettime() {
        $.ajax({
            url: "/time",
            timeout: 10000, //超时时间设置为10秒;
            success: function (data) {
                $("#tim").html(data)
            },
            error: function (xhr, type, errorThrown) {
    
            }
        });
    }
    
    function get_c1_data() {
        $.ajax({
            url: "/c1",
            success: function (data) {
                $(".num h1").eq(0).text(data.confirm);
                console.log(data.confirm);
                $(".num h1").eq(1).text(data.suspect);
                $(".num h1").eq(2).text(data.heal);
                $(".num h1").eq(3).text(data.dead);
            },
            error: function (xhr, type, errorThrown) {
    
            }
        })
    }
    
    function get_c2_data() {
        $.ajax({
            url: "/c2",
            success: function (data) {
                ec_center_option.series[0].data = data.data
                ec_center.setOption(ec_center_option)
            },
            error: function (xhr, type, errorThrown) {
    
            }
        })
    }
    
    function get_l1_data() {
        $.ajax({
            url: "/l1",
            success: function (data) {
                var day = []
                for (let i = 0; i < 8; i++) {
                    day[7 - i] = data.day[data.day.length - 1 - i]
                }
                // ec_left1_Option.xAxis[0].data=data.day
                ec_left1_Option.xAxis[0].data = day
                ec_left1_Option.series[0].data = data.confirm
                console.log(data.confirm);
                ec_left1_Option.series[1].data = data.suspect
                // ec_left1_Option.series[2].data = data.heal
                // ec_left1_Option.series[3].data = data.dead
                ec_left1.setOption(ec_left1_Option)
                // console.log(day)
            },
            error: function (xhr, type, errorThrown) {
    
            }
        })
    }
    
    function get_l2_data() {
        $.ajax({
            url: "/l2",
            success: function (data) {
                var day = []
                for (let i = 0; i < 8; i++) {
                    day[7 - i] = data.day[data.day.length - 1 - i]
                }
                ec_left2_Option.xAxis[0].data = day
                ec_left2_Option.series[0].data = data.confirm_add
                ec_left2_Option.series[1].data = data.suspect_add
                ec_left2.setOption(ec_left2_Option)
            },
            error: function (xhr, type, errorThrown) {
    
            }
        })
    }
    
    function get_r1_data() {
        $.ajax({
            url: "/r1",
            success: function (data) {
                ec_right1_option.xAxis.data = data.city;
                ec_right1_option.series[0].data = data.confirm;
                ec_right1.setOption(ec_right1_option);
            }
        })
    }
    
    function get_r2_data() {
        $.ajax({
            url: "/r2",
            success: function (data) {
                ec_right2_option.series[0].data = data.data
                ec_right2.setOption(ec_right2_option)
    
            },
            error: function (xhr, type, errorThrown) {
    
            }
        })
    }
    
    gettime()
    get_c1_data()
    get_c2_data()
    get_l1_data()
    get_l2_data()
    get_r1_data()
    get_r2_data()
    
    setInterval(gettime, 1000)
    setInterval(get_c1_data, 1000 * 10)
    setInterval(get_c2_data, 10000 * 10)
    setInterval(get_l1_data, 10000 * 10)
    setInterval(get_l2_data, 10000 * 10)
    setInterval(get_r1_data, 10000 * 10)
    setInterval(get_r2_data, 10000 * 10)
\end{lstlisting}

\section{ec\underline{~}center.js}
\begin{lstlisting}
    var ec_center = echarts.init(document.getElementById('c2'), "dark");

    var mydata = []
    
    var ec_center_option = {
        title: {
            text: '全国累计确诊地图',
            subtext: '',
            x: 'center'
        },
        tooltip: {
            trigger: 'item'
        },
        //左侧小导航图标
        visualMap: {
            show: true,
            x: 'left',
            y: 'bottom',
            textStyle: {
                fontSize: 8,
            },
            splitList: [{ start: 1,end: 9 },
                {start: 10, end: 99 }, 
                { start: 100, end: 999 },
                {  start: 1000, end: 9999 },
                { start: 10000 }],
            color: ['#8A3310', '#C64918', '#E55B25', '#F2AD92', '#F9DCD1']
        },
        //配置属性
        series: [{
            name: '累计确诊人数',
            type: 'map',
            mapType: 'china',
            roam: false, //拖动和缩放
            itemStyle: {
                normal: {
                    borderWidth: .5, //区域边框宽度
                    borderColor: '#009fe8', //区域边框颜色
                    areaColor: "#ffefd5", //区域颜色
                },
                emphasis: { //鼠标滑过地图高亮的相关设置
                    borderWidth: .5,
                    borderColor: '#4b0082',
                    areaColor: "#fff",
                }
            },
            label: {
                normal: {
                    show: true, //省份名称
                    fontSize: 8,
                },
                emphasis: {
                    show: true,
                    fontSize: 8,
                }
            },
            data:[] //mydata //数据
        }]
    };
    ec_center.setOption(ec_center_option)
\end{lstlisting}


