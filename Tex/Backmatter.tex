%---------------------------------------------------------------------------%
%->> Backmatter
%---------------------------------------------------------------------------%
\chapter[致谢]{致\quad 谢}\chaptermark{致\quad 谢}% syntax: \chapter[目录]{标题}\chaptermark{页眉}
%\thispagestyle{noheaderstyle}% 如果需要移除当前页的页眉
%\pagestyle{noheaderstyle}% 如果需要移除整章的页眉

行文至此,落笔为念。两年半前踏入大学的之日仿佛就在昨天,目光所到之处,每一间教室,每一个角落,皆是回忆。我将永远热爱那段在机房调试代码的时光,每一行代码,每一个数据都犹如眼睛,所到之处里尽是无限的灵感。

首先在此我要感谢我的导师王震教授,感谢他在本文写作过程中给出的意见与格式修改的建议。从开题到初稿再到定稿,老师都在给予无微不至指导与帮助。在老师身上我学到很多东西,对学术认真、一丝不苟的精神。这些收获将会令我受益终生,让我在未来的学习与工作中多一份信心与勇气。

借此机会,特别感谢我的父母。二十余载求学之路,离不开背后父母的默默付出。在我面对困难时,您永远是我最坚强的后盾,不断支持着我去追寻我所喜爱的事业。

除此之外,我还要提到一位特别重要的人——姚庆悦。青春兵荒马乱,我们潦草的离散,感谢你曾出现在我的生命中并陪我走过那段难忘的高中时光。

本文使用\LaTeX{}模版ucasthesis写作,在此特别感谢casthesis作者吴凌云学长,gbt7714-bibtex-style开发者zepinglee和ctex众多开发者们。尤其感谢GitHub项目ucasthesis作者莫晃锐的开源\LaTeX{}模版。在此对所有\LaTeX{}社区的贡献者表示感谢!

% \chapter{作者简历及攻读学位期间发表的学术论文与研究成果}

% \textbf{本科生无需此部分}。

% \section*{作者简历:}

% \subsection*{casthesis作者}

% 吴凌云,福建省屏南县人,中国科学院数学与系统科学研究院博士研究生。

% \subsection*{ucasthesis作者}

% 莫晃锐,湖南省湘潭县人,中国科学院力学研究所硕士研究生。

% \section*{已发表(或正式接受)的学术论文:}

% {
% \setlist[enumerate]{}% restore default behavior
% \begin{enumerate}[nosep]
%     \item ucasthesis: A LaTeX Thesis Template for the University of Chinese Academy of Sciences, 2014.
% \end{enumerate}
% }

% \section*{申请或已获得的专利:}

% (无专利时此项不必列出)

% \section*{参加的研究项目及获奖情况:}

% 可以随意添加新的条目或是结构。

\cleardoublepage[plain]% 让文档总是结束于偶数页,可根据需要设定页眉页脚样式,如 [noheaderstyle]
%---------------------------------------------------------------------------%
